\RequirePackage[l2tabu,orthodox]{nag}
\pdfoutput=1
\pdfminorversion=3
\documentclass[compress,red]{beamer}\usetheme{Warsaw}\useoutertheme[subsection=false]{smoothbars}
\setbeamertemplate{navigation symbols}{}%remove navigation symbols
\hypersetup{pdfpagemode=FullScreen}  
\usepackage[english]{babel}
\usepackage{times}
\usepackage{amsmath}
\usepackage{amssymb}
\usepackage{latexsym}
\usepackage{graphicx}
\usepackage{url}
\usepackage{listings}
%\usepackage[hyperfootnotes=false,hidelinks=true]{hyperref}
\usepackage{multirow}
\usepackage{subfigure} 
%\usepackage{algorithm}
\usepackage{algorithmic}
\usepackage{hyperref}

\title{Practical Tools to Pursue Open Science}
\author{Peter Wittek}
\institute{ICFO-The Institute of Photonic Sciences\\
and\\
University of Bor{\aa}s}
\date{4 June 2015}
\subject{Talks}
 
% Delete this, if you do not want the table of contents to pop up at
% the beginning of each subsection:
\AtBeginSubsection[]
{
  \begin{frame}<beamer>{Outline}
    \tableofcontents[currentsection,currentsubsection]
  \end{frame}
}

%\beamerdefaultoverlayspecification{<+->}
\begin{document}

\frame[plain]{
  \titlepage
}

%%%%%%%%%%%%%%%%%%%%%%%%%%%%%%%%%%%%%%%%%%%%%%%%%%%%%%%%%%%%%%%%%%%%%%%

%%%%%%%%%%%%%%%%%%%%%%%%%%%%%%%%%%%%%%%%%%%%%%%%%%%%%%%%%%%%%%%%%%%%%%%
\begin{frame}{What Is Open Science?}
\begin{itemize}
\item Accessible dissemination of all scientific output.
\item Many aspects:
\begin{itemize}
  \item Open access.
  \item Open data.
  \item Open methodology.
  \item Open source code.
  \item Networked science.
\end{itemize}
\item We still do publishing the way we did 200 years ago.
\item \href{http://www.theguardian.com/science/2012/jan/16/academic-publishers-enemies-science}{Academic publishers have become the enemies of science}
\end{itemize}
\end{frame}

%%%%%%%%%%%%%%%%%%%%%%%%%%%%%%%%%%%%%%%%%%%%%%%%%%%%%%%%%%%%%%%%%%%%%%%
\begin{frame}{Why Open Science?}
\begin{itemize}
\item Shift focus from paper-based publishing.
\begin{itemize}
  \item Papers are leaps, whereas sciences advances in small steps.
\end{itemize}
\item Make other people's life easier by lowering the barrier to entry.
\item Great for early-stage researchers to build up reputation.
\begin{itemize}
  \item It is also easy to do bad science.
\end{itemize}
\item Getting scooped? Drawing attention is actually hard.
\end{itemize}
\end{frame}

%%%%%%%%%%%%%%%%%%%%%%%%%%%%%%%%%%%%%%%%%%%%%%%%%%%%%%%%%%%%%%%%%%%%%%%
\begin{frame}{A Plethora of Options}
\begin{itemize}
\item For papers, the standard in this community is arXiv.
\begin{itemize}
  \item See self-archiving policies.
  \item Open access journals.
\end{itemize}
\item Social networking: \href{https://www.academia.edu/}{Academia.edu}, \href{https://www.researchgate.net/}{ResearchGate}, \href{http://scholar.google.com/}{Google Scholar}
\item Sharing other output: \href{http://figshare.com/}{FigShare}, \href{http://www.slideshare.net/}{SlideShare}, \href{https://github.com/}{GitHub}
\item Q\&A sites: WhateverOverflow (\href{http://mathoverflow.net/}{Maths}, \href{http://www.physicsoverflow.org/}{Physics}, \href{http://stackoverflow.com/}{Programmers}, \href{http://gardening.stackexchange.com/}{
Gardening \& Landscaping})
\end{itemize}
\end{frame}

%%%%%%%%%%%%%%%%%%%%%%%%%%%%%%%%%%%%%%%%%%%%%%%%%%%%%%%%%%%%%%%%%%%%%%%
\begin{frame}{GitHub}
\begin{itemize}

\item Became the de facto open source site.

\item Quick way to put things online: code, notebooks, figures, data, blog, website.

\item Ease of collaboration: other people can build on your work with low effort.

\item Stars, followers, forks: good work will attract attention.

\item Gaining features designed especially for researchers.
\begin{itemize}
  \item Citable code (DOI).
  \item Notebooks render almost nicely.
\end{itemize}

\item Reflects scientific thinking and workflow a lot better than social networks designed for researchers.
\end{itemize}
\end{frame}

%%%%%%%%%%%%%%%%%%%%%%%%%%%%%%%%%%%%%%%%%%%%%%%%%%%%%%%%%%%%%%%%%%%%%%%
\begin{frame}[fragile]{Version Control Systems}
\begin{itemize}
\item Do not do damage to yourself: start using a version control system now.

\item Think of it as an annotated history of whatever you work on.

\item Avoid two common problems:
 \begin{itemize}
\item One million versions of some file you work on: {\scriptsize \verb+Untitled245version16_gl_edited_version5_with_comments2.docx+}

\item Limited undo.
\end{itemize}
\item What they are bad at: binary files and large files.
\end{itemize}
\end{frame}

%%%%%%%%%%%%%%%%%%%%%%%%%%%%%%%%%%%%%%%%%%%%%%%%%%%%%%%%%%%%%%%%%%%%%%%
\begin{frame}{Workflow in Git}
\begin{itemize}
\item Git emerged as a new paradigm to VCS: distributed repositories.
 
\item You will probably use a total of five or six git commands.
\begin{itemize}
  \item Initialize a repository.
  \item Add files.
  \item Make commits.
  \item Check out earlier versions or other branches.
  \item Push to and pull from GitHub.
  \item Maybe synchronize with upstream.
\end{itemize}
\end{itemize}
\end{frame}

%%%%%%%%%%%%%%%%%%%%%%%%%%%%%%%%%%%%%%%%%%%%%%%%%%%%%%%%%%%%%%%%%%%%%%%
\begin{frame}{Sharing and Collaborating}
\begin{itemize}
\item Literate programming: notebooks.
\begin{itemize}
  \item Mix text, mathematical formulas, code, figures and data in the same context.
  \item It is outstanding for explaining ideas to others, it can serve as a computational appendix.
  \item It is not so good for development.
\end{itemize}
\item Code: how actual development is done. It helps others to continue your work.
\end{itemize}
\end{frame}

%%%%%%%%%%%%%%%%%%%%%%%%%%%%%%%%%%%%%%%%%%%%%%%%%%%%%%%%%%%%%%%%%%%%%%%
\begin{frame}{Blog-Aware Static Website}
\begin{itemize}
  \item Static website: secure and fast.
  \item Recommended: \href{http://blog.getpelican.com/}{Pelican}
  \begin{itemize}
     \item Written in Python.
     \item Takes 2 minutes to create a decent-looking website.
     \item Mathematical formulas rendered nicely.
     \item Can integrate Jupyter notebooks.
  \end{itemize}
  \item Hosting on GitHub: username.github.io
\end{itemize}
\end{frame}

%%%%%%%%%%%%%%%%%%%%%%%%%%%%%%%%%%%%%%%%%%%%%%%%%%%%%%%%%%%%%%%%%%%%%%%
\begin{frame}{Links}
\begin{itemize}
\item \href{https://github.com/blog/1840-improving-github-for-science}{Improving GitHub for science}

\item \href{https://github.com/ipython/ipython/wiki/A-gallery-of-interesting-IPython-Notebooks\#reproducible-academic-publications}{Reproducible academic publications in Jupyter notebooks}

\item \href{https://github.com/showcases/science}{Showcase of science projects on GitHub}

\item \href{http://nyuccl.org/pages/gittutorial/}{Git for scientists}

\item \href{https://github.com/mnielsen}{Michael Nielsen's GitHub page}

\item \href{http://www.nature.com/news/programming-pick-up-python-1.16833}{Python promoted in Nature}

\item \href{http://icons.icfo.eu/}{Don't forget to visit ICONS webpage!!}
\end{itemize}
\end{frame}

\end{document}
