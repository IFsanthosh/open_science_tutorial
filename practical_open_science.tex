\RequirePackage[l2tabu,orthodox]{nag}
\pdfoutput=1
\pdfminorversion=3
\documentclass[compress,red]{beamer}\usetheme{Warsaw}\useoutertheme[subsection=false]{smoothbars}
\setbeamertemplate{navigation symbols}{}%remove navigation symbols
\hypersetup{pdfpagemode=FullScreen}  
\usepackage[english]{babel}
\usepackage{times}
\usepackage{amsmath}
\usepackage{amssymb}
\usepackage{latexsym}
\usepackage{graphicx}
\usepackage{url}
\usepackage{listings}
%\usepackage[hyperfootnotes=false,hidelinks=true]{hyperref}
\usepackage{multirow}
\usepackage{subfigure} 
%\usepackage{algorithm}
\usepackage{algorithmic}
\usepackage{hyperref}

\title{Practical Tools to Pursue Open Science}
\author{Peter Wittek}
\institute{ICFO-The Institute of Photonic Sciences\\
and\\
University of Bor{\aa}s}
\date{4 June 2015}
\subject{Talks}
 
% Delete this, if you do not want the table of contents to pop up at
% the beginning of each subsection:
\AtBeginSubsection[]
{
  \begin{frame}<beamer>{Outline}
    \tableofcontents[currentsection,currentsubsection]
  \end{frame}
}

%\beamerdefaultoverlayspecification{<+->}
\begin{document}

\frame[plain]{
  \titlepage
}

%%%%%%%%%%%%%%%%%%%%%%%%%%%%%%%%%%%%%%%%%%%%%%%%%%%%%%%%%%%%%%%%%%%%%%%

%%%%%%%%%%%%%%%%%%%%%%%%%%%%%%%%%%%%%%%%%%%%%%%%%%%%%%%%%%%%%%%%%%%%%%%
\begin{frame}{What Is Open Science?}
\begin{itemize}
\item \ldots
\end{itemize}
\end{frame}

%%%%%%%%%%%%%%%%%%%%%%%%%%%%%%%%%%%%%%%%%%%%%%%%%%%%%%%%%%%%%%%%%%%%%%%
\begin{frame}{Why Open Science?}
\begin{itemize}
\item \ldots
\end{itemize}
\end{frame}

%%%%%%%%%%%%%%%%%%%%%%%%%%%%%%%%%%%%%%%%%%%%%%%%%%%%%%%%%%%%%%%%%%%%%%%
\begin{frame}{A Plethora of Options}
\begin{itemize}
\item \ldots
\end{itemize}
\end{frame}

%%%%%%%%%%%%%%%%%%%%%%%%%%%%%%%%%%%%%%%%%%%%%%%%%%%%%%%%%%%%%%%%%%%%%%%
\begin{frame}{GitHub}
\begin{itemize}
\item Quick way to put things online: code, notebooks, figures, data, blog, website.

\item Ease of collaboration: other people can build on your work with low effort.

\item Stars, followers, forks: good work will attract attention.

\item Gaining features designed especially for researchers.

\item Became the de facto open source site.

\end{itemize}
\end{frame}

%%%%%%%%%%%%%%%%%%%%%%%%%%%%%%%%%%%%%%%%%%%%%%%%%%%%%%%%%%%%%%%%%%%%%%%
\begin{frame}[fragile]{Version Control Systems}
\begin{itemize}
\item Do not do damage to yourself: start using a version control system now.

\item Think of it as an annotated history of whatever you work on.

\item Avoid two common problems:
 \begin{itemize}
\item One million versions of some file you work on: {\scriptsize \verb+Untitled245version16_gl_edited_version5_with_comments2.docx+}

\item Limited undo.
\end{itemize}
\item What they are bad at: binary files and large files.
\end{itemize}
\end{frame}

%%%%%%%%%%%%%%%%%%%%%%%%%%%%%%%%%%%%%%%%%%%%%%%%%%%%%%%%%%%%%%%%%%%%%%%
\begin{frame}{Workflow in Git}
\begin{itemize}
\item Git emerged as a new paradigm to VCS: distributed repositories.
 
\item You will probably use a total of five or six git commands.
\begin{itemize}
  \item Initialize a repository.
  \item Add files.
  \item Make commits.
  \item Check out earlier versions or other branches.
  \item Push to and pull from GitHub.
  \item Maybe synchronize with upstream.
\end{itemize}
\end{itemize}
\end{frame}

%%%%%%%%%%%%%%%%%%%%%%%%%%%%%%%%%%%%%%%%%%%%%%%%%%%%%%%%%%%%%%%%%%%%%%%
\begin{frame}{Sharing and Collaborating}
\begin{itemize}
\item Code
\item Notebooks
\end{itemize}
\end{frame}

%%%%%%%%%%%%%%%%%%%%%%%%%%%%%%%%%%%%%%%%%%%%%%%%%%%%%%%%%%%%%%%%%%%%%%%
\begin{frame}{Blog-aware Static Website}
\begin{itemize}
\item \ldots
\end{itemize}
\end{frame}

\end{document}
